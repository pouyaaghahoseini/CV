%%%%%%%%%%%%%%%%%%%%%%%%%%%%%%%%%%%%%%%%%
% Medium Length Professional CV
% LaTeX Template
% Version 2.0 (8/5/13)
%
% This template has been downloaded from:
% http://www.LaTeXTemplates.com
%
% Original author:
% Trey Hunner (http://www.treyhunner.com/)
%
% Important note:
% This template requires the resume.cls file to be in the same directory as the
% .tex file. The resume.cls file provides the resume style used for structuring the
% document.
%
%%%%%%%%%%%%%%%%%%%%%%%%%%%%%%%%%%%%%%%%%

%----------------------------------------------------------------------------------------
%	PACKAGES AND OTHER DOCUMENT CONFIGURATIONS
%----------------------------------------------------------------------------------------

\documentclass{resume} % Use the custom resume.cls style
\usepackage{bookmark}
\usepackage{hyperref}
\usepackage{xcolor}
\usepackage[left=0.75in,top=0.6in,right=0.75in,bottom=0.6in]{geometry} % Document margins

\name{Pouya Aghahoseini} % Your name
\address{424 Hafez Ave, Tehran, Iran, 15875-4413. \\ (+98)9102117201 }
\address{ \href{mailto:pouyaaghahoseini@gmail.com}{pouyaaghahoseini@gmail.com} \\ \href{https://github.com/pouyaaghahoseini}{GitHub} \\ \href{https://linkedin.com/in/pouyaaghahoseini}{LinkedIn}\\
	\href{http://pouyaaghahoseini.gitlab.io}{Website} }

\begin{document}
	
	%----------------------------------------------------------------------------------------
	%	EDUCATION SECTION
	%----------------------------------------------------------------------------------------
	\chapter{Educatio}
	\section{Education}
	\begin{rSection}{Education}
		
		{\bf Bachelor of Science in Computer Science} \hfill {\em 2016-2020(Expected)} \\ 
		\href{https://www.topuniversities.com/universities/amirkabir-university-technology}{Amirkabir University of Technology, Tehran, Iran}\\
		Cumulative GPA: 3.6/4 (17.28/20)\\
		Last Two Years GPA(68 Credits): 3.9/4 (18.37/20)\\
		Last Year GPA(32 Credits) : 4/4 (19.55/20)\\
		Member of Student Scientific Chapter (Head of Informatics) \\
		
		{\bf Diploma in Physics and Mathematics Discipline} \hfill {\em 2012-2016}\\ 
		Allameh Helli High School, Tehran , Iran\\
		National Organization for Development of Exceptional Talents (NODET)\\
		Two Science Seminar Projects (Physics and Mechanics Fields)\\
		GPA: 18.5/20\\
		
	\end{rSection}
	\section{Test Scores}
	\begin{rSection}{Test Scores}
		\begin{itemize}{\bfseries}
			\item {\bfseries TOEFL: 103 (R:28 L:27 S:27 W:21)}
			\item {\bfseries GRE: 321 (V:158 Q:163 AWA:3.5)}
		\end{itemize}
		
	\end{rSection}
	
	%----------------------------------------------------------------------------------------
	%	WORK EXPERIENCE SECTION
	%----------------------------------------------------------------------------------------
	\section{Experience}
	\begin{rSection}{Experience}
		\begin{rSubsection}{Lead Teaching Assistantship, Fundamentals of Programming: C Language}{Present}{Instructor: Dr.Farzaneh Saalaari}{Amirkabir University of Technology}
		\item Holding Weekly Workshops
		\item Designing and Evaluating Assignments
		
		\end{rSubsection}					%------------------------------------------------
		\begin{rSubsection}{Teaching Assistantship, Compilers}{Present}{Instructor: Dr.Mohammad Hassan Shirali-Shahreza}{Amirkabir University of Technology}
		\item Exam Preperation Classes and Assignment Evaluation
		\end{rSubsection}					%------------------------------------------------
		\begin{rSubsection}{Teaching Assistantship, Data Structures and Algorithms }{Feb 2019 - Jun 2019}{Instructor: Dr.Mohammad Akbari}{Amirkabir University of Technology}
			\item 
			Held weekly classes and office hours.
			\item 
			Designed and Evaluated assignment and projects.
		\end{rSubsection}
		%------------------------------------------------
		\begin{rSubsection}{Winter School Organizing Committee Member}{Mar 2019}{11th Winter School on Computational Geometry}{Amirkabir University of Technology}
			\item 	The Winter School on Computational Geometry is held every year by Math and Computer Science Faculty, in collaboration with IPM Institute for Research in Fundamental Sciences.
			\item Notable researchers in computational geometry are invited for seminars and workshops in this program.
		\end{rSubsection}
		
		
		%------------------------------------------------
		
		\begin{rSubsection}{Lead Teaching Assistantship, Introduction to the Theory of Computation}{Sep 2018 - Jan 2019}{Instructor: Dr.Fatemeh Zare-Mirakabad}{Amirkabir University of Technology}
			\item Held weekly classes 
			\item Supervised monthly workshops
			\item Designed and Evaluated assignment 
			
		\end{rSubsection}
		
		%------------------------------------------------
		
		\begin{rSubsection}{Student Scientific Chapter Membership}{Mar 2018 - Mar  2019}{Head Of Informatics and Competitions}{Amirkabir University of Technology}
			\item Organized Data Science Seminars And Workshops.
			\item Coordinated 5 ACM Contests.
			\item Arranged On-Site Visits To Leading Tech Companies In Tehran.
			\item Arranged Talks With Alumni Currently Studying Or Working Abroad.
			\item Held 2018 New Student Orientation Program.
			\item Arranged Exam Preparation Classes(Calculus,Programming,Differential Equation).
		\end{rSubsection}
		
		
	\end{rSection}
	
	%----------------------------------------------------------------------------------------
	%	PROJECTS SECTION
	%---------------------------------------------------------------------------------------
	\section{Projects}
	\begin{rSection}{Projects}
		
		\begin{rSubsection}{\href{https://github.com/pouyaaghahoseini/System-Queueing-Simulation}{Factory Simulation with MATLAB}}{Sep 2018 - Jan 2019}{Simulation  Project}{Amirkabir University of Technology - Industrial Engineering Faculty}
			\item System modeling and simulation with MATLAB
			\item Simulation of a factory with 4 lines of work with different performances and various response time of each agent.
			
			
		\end{rSubsection}
	
		\begin{rSubsection}{\href{https://github.com/pouyaaghahoseini/Medify}{Medify: An Application for Fixing Tags And Organizing Media Files}}{Jan 2018 - July 2018}{Software Engineering Course Project}{Amirkabir University of Technology}
			
			\item The application's development was under the supervision of Dr.Amin Gheibi and was Developed using Scrum Methodology for Agile Software Development.
			\item The coursework required 5 presentations throughout the semester which included a 5-minute startup pitch presentation and the presentation of the first MVP(Minimum Viable Product).
			\item Weekly feedback Sessions.
			\item Scrum master role changed every two weeks.	
		\end{rSubsection}
		
		\begin{rSubsection}{Designing and Creating A Database From YAGO Datasets}{Jan 2018 - July 2018}{\href{https://github.com/pouyaaghahoseini/Database-Course}{Database Course Project}}{Amirkabir University of Technology}
			\item Data Extraction using TSV files acquired from \href{https://www.mpi-inf.mpg.de/departments/databases-and-information-systems/research/yago-naga/yago/downloads/}{max plank institute website}.
			\item Processing and Clustering TSV files with Python and MySQL.
			\item Designing and Creating A 3NF Database.
			
		\end{rSubsection}	
		
		
		\begin{rSubsection}{\href{https://github.com/pouyaaghahoseini/DS-Course/tree/master/SE-Skiplist}{SESSET: A Space Efficient Skiplist Data Structure}}{Sep 2017 - January 2018}{Data Structures Course Project}{Amirkabir University of Technology}
			\item SSET‌ was implemented with skiplist data structure.
			\item Each node consisted a BDeque which resulted in more efficient operations.
			\item All operations are done in $O(logn)^{E} + O(\sqrt{n})$ time.
		\end{rSubsection}
		
		
		\begin{rSubsection}{\href{https://github.com/pouyaaghahoseini/DS-Course}{Range-Tree: A 2-D version of Range-Tree Data Structure}}{Sep 2017 - January 2018}{Data Structures Course Project}{Amirkabir University of Technology}
			\item The Tree is primarily built with the input points.
			\item Queries are asked in the form of rectangle coordinates. 
			\item The implementation was in C++ language.
			
		\end{rSubsection}
		
		
		
	\end{rSection}
	%----------------------------------------------------------------------------------------
	%	HONORS AND AWARDS SECTION
	%----------------------------------------------------------------------------------------
	\section{Honors And Awards}
	\begin{rSection}{Honors and Awards}
		\begin{rSubsection}{Top 10\% in Graduating Class}{Present}{Math and Comptuer Science Faculty}{Amirkabir University of Technology}
			\item Class of 70 students
		\end{rSubsection}
		
		\begin{rSubsection}{Ranked $1^{st}$ GPA at Fall-2018 and Spring-2019 Semesters }{Sep 2018 - Jun 2019}{Math and Comptuer Science Faculty}{Amirkabir University of Technology}
			\item Class of 70 Students
		\end{rSubsection}	
		
		\begin{rSubsection}{One of the Top Teams at Amirkabir ACM-ICPC Competition}{Nov 2018}{Computer Engineering Faculty}{Amirkabir University of Technology}
			\item The ACM-ICPC Contest is held every year at Amirkabir University of Technology. Teams from all universities across the country take part in this competition.
			\item Candidates for representing the university in Asia Region contest are primarily filtered here.
		\end{rSubsection}
		
		\begin{rSubsection}{Ranked Within the Top 0.7\% (1,169 among 162,879 students) in Iranian University Entrance Exam}{Jul 2016}{Tehran, Iran }{ }
			\item The competition is intense since it is the only means to gain admission to universities.
		\end{rSubsection} 
		
		\begin{rSubsection}{Accepted in NODET (National Organization for Development of Exceptional Talents) High School Entrance Exam}{Jul 2012}{Tehran, Iran }{ }
			\item NODET student selection exam is held every year nationwide for students starting high school. The organization is responsible for a number of schools across the country, which train students on a more advanced level in each field of study.
		\end{rSubsection}
		
	\end{rSection}
	%----------------------------------------------------------------------------------------
	%	SKILLS SECTION
	%----------------------------------------------------------------------------------------
	\section{Skills}
	\begin{rSection}{SKILLS}
		
		\begin{tabular}{ @{} >{\bfseries}l @{\hspace{6ex}} l }
			Programming Languages & C/C++, Python, R, MATLAB \\
			Web Technologies & HTML, CSS, Javascript, MySQL \\
			Libraries & numPy, Pandas \\
			Operating Systems & Linux, Windows\\
			Languages & Persian, English \\
			Miscellaneous & \LaTeX, Jupyter, Microsoft Office, GAMS, Database Design\\
		\end{tabular}
		
	\end{rSection}
	
	%----------------------------------------------------------------------------------------
%	RESEARCH ‌INTEREST SECTION
%----------------------------------------------------------------------------------------	
	\section{Research Interests}
	\begin{rSection}{Research Interests}
		
		\begin{itemize}{\bfseries}
			\item {\bfseries Natural Language Processing}
			\item {\bfseries Machine Learning }
			\item {\bfseries Deep Learning }
			\item {\bfseries Data Science}
			\item {\bfseries Artificial Intelligence}
		\end{itemize}
		
	\end{rSection}
	
	%----------------------------------------------------------------------------------------
	
\end{document}
